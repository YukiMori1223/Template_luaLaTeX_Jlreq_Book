%フォント=======================================================================================
\usepackage[match,no-math]{luatexja-fontspec}
\setmainjfont[BoldFont=HaranoAjiMincho-Bold]{HaranoAjiMincho-Regular}
\setsansjfont[BoldFont=HaranoAjiGothic-Bold]{HaranoAjiGothic-Medium}
\setsansfont{TeX Gyre Heros}
\setmonofont{inconsolata}

% 見出し=======================================================================================
% 見出しは1, 1.1, 1.1.1, (a)まで
\setcounter{secnumdepth}{3}
% subsubsectionの番号を(a),(b),(c),...に
\renewcommand\thesubsubsection{(\alph{subsubsection})}
% sectionのデザイン
\ModifyHeading{section}{
	format={
			\begin{tcolorbox}[
					enhanced jigsaw,
					breakable=false,
					outer arc=0pt, arc=0pt,
					colback=white,
					leftrule=8pt, rightrule=0pt, toprule=0pt, bottomrule=1pt,
					left=3pt,right=0pt,top=0pt, bottom=0pt,
					colframe=darkgray,
				]
				\noindent #1 #2
			\end{tcolorbox}
		},
	lines=3,
	after_label_space=3pt
}

% ヘッダ・フッタ=======================================================================================
\NewPageStyle{mystyle}{
	nombre_position=bottom-center,
	running_head_position=top-right,
	odd_running_head=_section,
	running_head_ii_position=top-left,
	odd_running_head_ii=_chapter,
	mark_format={
			_section={\thesection #1},
			_chapter={第\thechapter 章 #1},
		},
	odd_head_format={\hss\underline{\makebox[\jlreqyokoheadlength][s]{#1}}\hss},
}
\pagestyle{mystyle}

%数式 =======================================================================================
\usepackage{amsthm,amsmath,amssymb,amsfonts}
\usepackage{siunitx}
\sisetup{
	group-digits= integer,
	group-separator= {,},
}
\numberwithin{equation}{chapter}

%図表 =======================================================================================
\usepackage{graphicx,xcolor}
\usepackage{float}
\usepackage{makecell}
\numberwithin{figure}{chapter}
\numberwithin{table}{chapter}

%リンク =======================================================================================
%hyperref
\usepackage[
	unicode,
	bookmarks=true,
	bookmarksdepth=2,
	bookmarksnumbered=true,
	setpagesize=false,
	pdfborder={0 0 0},
	pdftitle={},
	pdfsubject={},
	pdfauthor={}
]{hyperref}

%autorefの定義
\def\equationautorefname~#1\null{\textrm{~式(#1)\;}\null}
\def\figureautorefname~#1\null{図~#1\null}
\def\tableautorefname~#1\null{表~#1\null}
\def\partautorefname~#1\null{第~#1部\null}
\def\chapterautorefname~#1\null{第~#1章\null}
% \def\sectionautorefname~#1\null{第~#1節\null}
% \def\subsectionautorefname~#1\null{~#1節\null}
\def\pageautorefname~#1\null{~#1ページ \null}

%囲み枠 =======================================================================================
\usepackage{tcolorbox}
\tcbuselibrary{breakable, skins, theorems}

% 証明環境
\renewcommand{\proofname}{{【証明】}}

% 命題環境
\newtcbtheorem[auto counter, number within=chapter]{theorem}{命題}{
	enhanced, breakable, sharp corners,
	before skip=20pt,
	after skip=20pt,
	boxrule=1pt,
	colframe=black!80,
	colback=white,
	fonttitle=\bfseries\sffamily,
	colbacktitle =black!80,
	lefttitle=4mm,
	separator sign={:}
}{theorem}

% 定義環境
\newtcbtheorem[auto counter, number within=chapter]{definition}{定義}{
	enhanced, breakable, sharp corners,
	before skip=20pt,
	after skip=20pt,
	boxrule=1pt,
	colframe=black!50,
	colback=white,
	fonttitle=\bfseries\sffamily,
	colbacktitle =black!50,
	lefttitle=4mm,
	separator sign={:}
}{def}

%コラム環境
\newtcbtheorem[auto counter, number within=chapter]{column}{コラム}{
	enhanced, breakable,
	before skip=20pt,
	after skip=20pt,
	boxrule=1pt,
	colframe=black!50,
	colbacktitle =black!50,
	colback=white,
	fonttitle=\bfseries\sffamily,
	separator sign={:}
}{column}

% プログラム環境
\tcbuselibrary{listings,skins,breakable}
\let\theHFancyVerbLine\theFancyVerbLine
\def\theFancyVerbLine{\rmfamily\footnotesize{\arabic{FancyVerbLine}}}
\usepackage{shellesc}
\tcbuselibrary{minted}

\newtcblisting[auto counter, number within=chapter]{program}[3][text]{
	breakable, enhanced, listing only, sharp corners,
	before skip=20pt,
	after skip=20pt,
	boxrule=1pt,
	colframe=blue!50,
	colback=white,
	listing engine=minted,
	minted language=#1,
	minted style=vs,
	minted options={
			linenos,
			breaklines,
			autogobble,
			fontsize=\normalsize,
			numbersep=2mm,
			baselinestretch=1},
	overlay={
			\begin{tcbclipinterior}
				\fill[gray!25] (frame.south west) rectangle ([xshift=4mm]frame.north west);
			\end{tcbclipinterior}},
	title={プログラム \thetcbcounter : #2 },
	fonttitle=\bfseries\sffamily,
	colbacktitle =blue!50,
	lefttitle=4mm,
	label = program:#3,
}

\newtcblisting{program*}[1]{
	breakable, enhanced, listing only, sharp corners,
	before skip=20pt,
	after skip=20pt,
	boxrule=1pt,
	colframe=blue!50,
	colback=white,
	listing engine=minted,
	minted language=#1,
	minted style=vs,
	minted options={
			linenos,
			breaklines,
			autogobble,
			fontsize=\normalsize,
			numbersep=2mm,
			baselinestretch=1},
	overlay={
			\begin{tcbclipinterior}
				\fill[gray!25] (frame.south west) rectangle ([xshift=4mm]frame.north west);
			\end{tcbclipinterior}
		}
}

\newtcblisting[auto counter, number within=chapter]{console}[2]{
	breakable, enhanced, listing only, sharp corners,
	before skip=20pt,
	after skip=20pt,
	boxrule=1pt,
	colframe=black!50,
	colback=black,
	coltext=white,
	title={出力 \thetcbcounter : #1 },
	fonttitle=\bfseries\sffamily,
	colbacktitle =black!50,
	lefttitle=4mm,
	label = console:#2
}

\newtcblisting{console*}{
	breakable, enhanced, listing only, sharp corners,
	before skip=20pt,
	after skip=20pt,
	boxrule=0pt,
	colframe=black!50,
	colback=black,
	coltext=white
}

% 索引 =======================================================================================
\usepackage{needspace}
\usepackage{makeidx}\makeindex
\usepackage{multicol}
\makeatletter
\renewenvironment{theindex}{%
	\columnsep2\zw
	\columnseprule\z@
	\begin{multicols}{2}[\section*{\indexname}]%
		\small
		\@mkboth{\indexname}{\indexname}%
		\parindent\z@
		\parskip\z@
		\lineskiplimit\z@
		\lineskip\z@
		\raggedbottom
		\raggedcolumns
		\raggedright
		\let\item\@idxitem}
		{\end{multicols}
	\clearpage}

\renewcommand{\@idxitem}{\par\leavevmode\hangindent1\zw\inhibitglue}
\renewcommand{\subitem}{\@idxitem\hangindent2\zw\hspace*{1\zw}\inhibitglue}
\renewcommand{\subsubitem}{\@idxitem\hangindent3\zw\hspace*{2\zw}\inhibitglue}
\renewcommand{\indexspace}{\par\vskip\baselineskip\relax}

\newcommand\idxdelim{\nobreak{\reset@font\scriptsize\space
		\leaders\hbox to .3333\zw{\hss\hbox{.}\hss}\hfill\space}\nobreak}

\newcommand*{\symbolindexname}{記号・数字}

\def\makeidxhead#1{%
	\needspace{2\baselineskip}%
	\vspace{\baselineskip}%
	\hbox to \columnwidth{\hfil
		\normalsize\sffamily\bfseries
		-\hspace{.5\zw}#1\hspace{.5\zw}-\hfil}\par
	\nopagebreak
}
\makeatother
